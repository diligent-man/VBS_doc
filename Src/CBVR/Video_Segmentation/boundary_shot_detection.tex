\section{CBVR step by step}
\subsection{Video Segmentation}
\subsubsection{Boundary shot detection}
\textbf{Syn}: Shot transition detection, cut detection\\

\noindent\textbf{Definition}\\
\indent Boundary shot detection is a video processing that automatically detect the transition between shots in the digital video with the aim of temporally segmenting a video into key frames. In the context of boundary shot detection, we have 2 different shot transition types:
\begin{figure}[h!]
	\centering
	\img{7cm}{5cm}{Picture/CBVR/Video_Segmentation/shot_transition_types.png}
	\caption{Shot transition types}
\end{figure}

\noindent\textbf{Approaches}
\noindent\textbf{Pixel-Based Methods}\\
In this method, we perform pixel-by-pixel comparison b/t two consecutive frames , and the cut is decided when it's higher than user-defined threshold.\\

\noindent Drawbacks:\vspace{-2.5mm}
\begin{enumerate}
	\item Manual threshold setting
	\item Sensitive to fast obj, cam movement \& zooming
\end{enumerate}


\noindent\textbf{Histogram-Based Methods}\\
In this method, we perform comparison b/t two consecutive frames' histograms via distance similarity, and the cut is decided when it's higher than user-defined threshold. There are several ways of calculating similarity distance: Manhattan distance, Euclidean distance, etc. This method is based on classical algorithms like SVD, HOG, HSV histogram, ...\\


\noindent\textbf{Edge-Based Methods}\\
In this approach, a transition is declared when the edge locations of the current frame exhibit a large difference with the edges of previous frames.


\noindent\textbf{Motion-Based Methods}\\
See linear motion prediction method based on wavelet coefficients for more information.

\noindent\textbf{Deep Learning-Based Methods}\\
