\subsection{Related problems}
\subsubsection{Content-Based Video Retrieval - CBVR} % Finished
\textbf{Definition}\\
\indent Content-based Video Retrieval refers to the provision of search facilities over archives of digital video
content, where these search facilities are based on
the outcome of an analysis of digital video content
to extract indexable data for the search process. CBVR includes \textbf{video segmentation}, and \textbf{low-level features extraction}. \textbf{Video segmentation} includes shot boundary detection, key frame extraction, and scene segmentation. \textbf{Feature extraction} includes extracting static features of key frames, object features, and motion features\\

\noindent\textbf{Unit of retrieval}\vspace{-.25cm}
\begin{enumerate}
	\item Shot\\
	Def 1: A shot lasts from the first moment the camera records until it’s stopped. (from Content-Based Video Retrieval in \href{https://link.springer.com/referencework/10.1007/978-0-387-39940-9}{Encyclopedia of database systems})\\
	
	Def 2: Series of continuous frames (images) from a single cam. (from \href{https://www.wevideo.com/blog/scene-vs-shot}{here})\\
	
	Def 3: A shot is a sequence of frames shot uninterruptedly by one camera (Wiki)\\
	-> Shot focuses on technical aspect of recording.
	
	\item Scene\\
	Def 1: A logical combination of video shots
	that together comprise some meaningful semantic
	unit (from Content-Based Video Retrieval in \href{https://link.springer.com/referencework/10.1007/978-0-387-39940-9}{Encyclopedia of database systems} )
	
	Def 2: A 'scene' is a series of shots combined to represent a continuous action or event. Scenes set up larger plot points and themes. Shots make scenes possible (from \href{https://www.wevideo.com/blog/scene-vs-shot}{here})\\
	-> Scene emphasizes on \textbf{narrative coherence}.
\end{enumerate}

\begin{figure}[h!]
	\centering
	\img{7cm}{5cm}{Picture/Overview/video_scene_shot_frame.png}
	\caption{Video Hierarchically}
\end{figure}

\noindent In many CBVR systems, \textbf{shot is more preferable} due to the fact that it is relatively easy to split a video file into its constituent shot in an automatic process called \textbf{shot boundary detection}\\

\noindent\textbf{UI/ UX}\vspace{-2.5mm}
UI should be able to
\begin{enumerate}
	\item represent digital video effectively
	\item represent all search facilities and other related components
	\item other misc components if have
\end{enumerate}

\noindent\textbf{CBVR with textual sources}\\
There are some text surrogate can be leveraged:\vspace{-2.5mm}
\begin{enumerate}
	\item Audio from video (spoken): leveraging text2speech for processing
	\item In-video text (written): leveraging OCR for processing
	\item Closed caption annotation \& metadata (e.g. MPEG 7, ...)
\end{enumerate}

\noindent\textbf{CBVR with visual sources}\\
\indent Low-level feature extraction is operated over individual key frames or scenes, but it leaves a huge semantic gap. To help bridge this semantic gap, a more complex semantic visual concepts have been introduced, such as \red{people} (faces, newsreaders), \red{location} (indoor/outdoor, cityscape/landscape), \red{objects} (buildings, cars, airplanes), \red{events} (explosions, violence) and \red{production techniques} such as camera motion.\\

\noindent\textbf{CBVR with audio sources}\\
\noindent Apart from speech2text, audio sources can be utilized in the way such that\vspace{-2.5mm}
\begin{enumerate}
	\item Non-standard or abnormal audio events, namely a window breaking, a door hard slam, and so on.
	
	\item Key events such as goal-mouth detection or onscreen scoreboard changing in soccer
\end{enumerate}
\begin{figure}[h!]
	\centering
	\img{10cm}{8cm}{Picture/Overview/CBVR_workflow.png}
	\caption{CBVR workflow (from Fı́schlár system at
		TRECVid in 2004)}
\end{figure}
%%%%%%%%%%%%%%%%%%%%%%%%%%%%%%%%%%%%%%%%%%%%%%%%%%%%


\subsubsection{Semantic-Based Video Retrieval - SBVR} % Incomplete
Discuss later


